% Tipo di documento. L'uso di twoside implica che i capitoli inizino sempre con la prima pagina a sinistra, eventualmente lasciando una pagina vuota nel capitolo precedente. Se questa cosa non va bene, è possibile rimuoverlo. 
\documentclass[a4paper, twoside,openright]{report}
% Dimensione dei margini
\usepackage[a4paper,top=3cm,bottom=3cm,left=3cm,right=3cm]{geometry} 
% Dimensione del font
\usepackage[fontsize=13pt]{scrextend}

% Lingua del testo
\usepackage[english,italian]{babel}
% Lingua per la bibliografia
\usepackage[fixlanguage]{babelbib}
% Codifica del testo
\usepackage[utf8]{inputenc} 
% Encoding del testo
% \usepackage[T1]{fontenc}
% Permette di generare testo fittizio. Mi è stato utile 
% per capire quale sarebbe stata l'impostazione del 
% testo nella pagina prima che scrivessi un determinato paragrafo
% \usepackage{lipsum}
% Per ruotare le immagini
\usepackage{rotating}
% Per modificare l'header delle pagine 
\usepackage{fancyhdr}               

% Librerie matematiche
\usepackage{amssymb}
\usepackage{amsmath}
\usepackage{amsthm}         

% Uso delle immagini
\usepackage{graphicx}
% Uso dei colori
\usepackage[dvipsnames]{xcolor}         
% Uso dei listing per il codice
\usepackage{listings}    
\usepackage{nameref}

% Definizioni per JavaScript
\lstdefinelanguage{JavaScript}{
  keywords={break, case, catch, continue, debugger, default, delete, do, else, false, finally, for, function, if, in, instanceof, new, null, return, switch, this, throw, true, try, typeof, var, void, while, with,let, const},
  morecomment=[l]{//},
  morecomment=[s]{/*}{*/},
  morestring=[b]',
  morestring=[b]",
  morestring=[b]`,
  ndkeywords={class, export, boolean, throw, implements, import, this},
  keywordstyle=\color{blue}\bfseries,
  ndkeywordstyle=\color{darkgray}\bfseries,
  identifierstyle=\color{black},
  commentstyle=\color{purple}\ttfamily,
  stringstyle=\color{red}\ttfamily,
  sensitive=true
}

\lstset{
   language=JavaScript,
   backgroundcolor=\color{white},
   extendedchars=true,
   basicstyle=\footnotesize\ttfamily,
   showstringspaces=false,
   showspaces=false,
   numbers=none,
   tabsize=2,
   breaklines=true,
   showtabs=false,
   captionpos=b,
   frame=single
}
% Per inserire gli hyperlinks tra i vari elementi del testo 
\usepackage{hyperref}     
% Diversi tipi di sottolineature
\usepackage[normalem]{ulem}

% -----------------------------------------------------------------

% Modifica lo stile dell'header
\pagestyle{fancy}
\fancyhf{}
\lhead{\rightmark}
\rhead{\textbf{\thepage}}
\fancyfoot{}
\setlength{\headheight}{12.5pt}

% Rimuove il numero di pagina all'inizio dei capitoli
\fancypagestyle{plain}{
  \fancyfoot{}
  \fancyhead{}
  \renewcommand{\headrulewidth}{0pt}
}

% Stile del codice
\lstdefinestyle{codeStyle}{
    % Colore dei commenti
    commentstyle=\color{teal},
    % Colore delle keyword
    keywordstyle=\color{Magenta},
    % Stile dei numeri di riga
    numberstyle=\tiny\color{gray},
    % Colore delle stringhe
    stringstyle=\color{violet},
    % Dimensione e stile del testo
    basicstyle=\ttfamily\footnotesize,
    % newline solo ai whitespaces
    breakatwhitespace=false,     
    % newline si/no
    breaklines=true,                 
    % Posizione della caption, top/bottom 
    captionpos=b,                    
    % Mantiene gli spazi nel codice, utile per l'indentazione
    keepspaces=true,                 
    % Dove visualizzare i numeri di linea
    numbers=left,                    
    % Distanza tra i numeri di linea
    numbersep=5pt,                  
    % Mostra gli spazi bianchi o meno
    showspaces=false,                
    % Mostra gli spazi bianchi nelle stringhe
    showstringspaces=false,
    % Mostra i tab
    showtabs=false,
    % Dimensione dei tab
    tabsize=2
} \lstset{style=codeStyle}

% Stile di codice per dimensioni maggiori, in cui ho avuto bisogno di un testo più piccolo (ad esempio se si vuole inserire del codice che ha linee molto lunghe). Per usare questo stile piuttosto che il precedente, usare 

% \lstset{style=longBlock}
%  % inserire il codice...
% \lstset{style=codeStyle}

% Il secondo comando consente di tornare allo stile precedente 
\lstdefinestyle{longBlock}{
    commentstyle=\color{teal},
    keywordstyle=\color{Magenta},
    numberstyle=\tiny\color{gray},
    stringstyle=\color{violet},
    basicstyle=\ttfamily\scriptsize,
    breakatwhitespace=false,         
    breaklines=true,                 
    captionpos=b,                    
    keepspaces=true,                 
    numbers=left,                    
    numbersep=5pt,                  
    showspaces=false,                
    showstringspaces=false,
    showtabs=false,                  
    tabsize=2
} \lstset{style=codeStyle}

% Togliendo il commento al comando che segue, si inseriscono nella bibliografia anche le fonti presenti in Bibliography.bib ma non citati direttamente con il comando \cite
% \nocite{*}

% Margini prima e dopo blocchi di codice, per avere più distanza
\lstset{aboveskip=20pt,belowskip=20pt}

% Modifica dello stile dei riferimenti, con il testo in nero
\hypersetup{
    colorlinks,
    linkcolor=black,
    citecolor=black
}

% Aggiunti definizioni, teoremi, linea e listing
\newtheorem{definition}{Definizione}[section]
\newtheorem{theorem}{Teorema}[section]
\providecommand*\definitionautorefname{Definizione}
\providecommand*\theoremautorefname{Teorema}
\providecommand*{\listingautorefname}{Listing}
\providecommand*\lstnumberautorefname{Linea}

\raggedbottom

%\newcommand{\cgs}[1]{{\textcolor{brown}[\textcolor{red}{\bf{GS: }}{ \textcolor{brown}{#1]}}}}             
%\newcommand{\cmc}[1]{{\textcolor{blue}[\textcolor{magenta}{\bf{MC: }}{ \textcolor{blue}{#1]}}}}

% ----------------------------------------------------------------

\setlength{\parskip}{\baselineskip}%
\setlength{\parindent}{0pt}%

% -----------------------------------------------------------------
% Comandi per il color coding delle correzioni
\newcommand{\torevise}[1]{\textcolor{red}{#1}}
\newcommand{\edit}[1]{\textcolor{NavyBlue}{#1}}
\newcommand{\approved}[1]{\textcolor{black}{#1}}

% Comandi per il color coding delle note e delle parti da sistemare
\newcommand{\tomodify}[1]{\textcolor{purple}{#1}}
\newcommand{\note}[1]{\textcolor{Bittersweet}{#1}}

% comandi per il codice inline
\newcommand{\jscode}[1]{\lstinline[language=JavaScript]{#1}}
\newcommand{\htmlcode}[1]{\lstinline[language=HTML]{#1}}

% Profondità della ToC
\setcounter{tocdepth}{2}

% Abbreviazioni per i comandi
\newcommand{\n}{\newline}
\newcommand{\nl}{\newline \newline }
% -----------------------------------------------------------------
\begin{document}


\begin{titlepage}
\begin{figure}[!htb]
    \centering
    \includegraphics[keepaspectratio=true,scale=0.5]{images/Frontespizio/cherubinFrontespizio.eps}
\end{figure}

\begin{center}
    \LARGE{UNIVERSITÀ DI PISA}
    \vspace{5mm}
    \\ \large{DIPARTIMENTO DI INFORMATICA}
    \vspace{5mm}
    \\ \LARGE{Laurea Triennale in Informatica}
\end{center}

\vspace{15mm}
\begin{center}
    {\LARGE{\bf Live LG:\\ \vspace{5mm} 
    app per sistemi IoT, Video e Smart TV di distribuzione contenuti multimediali targettizzati.
    }}
\end{center}
\vspace{30mm}

\begin{minipage}[t]{0.47\textwidth}
	{\large{Relatore:}{\normalsize\vspace{3mm}
	\bf\\ \large{Prof: Laura Semini} \normalsize\vspace{3mm}\bf \\ \large{Matteo Baldi}}}
\end{minipage}
\hfill
\begin{minipage}[t]{0.47\textwidth}\raggedleft
	{\large{Candidato:}{\normalsize\vspace{3mm} \bf\\ \large{Lorenzo Arcidiacono}}}
\end{minipage}

\vspace{30mm}
\hrulefill
\\\centering{\large{ANNO ACCADEMICO 2021/2022}}

\end{titlepage}
\include{chapters/Abstract}

\tableofcontents


\chapter{Introduzione}
\linespread{1.5}

\section{In Breve}
Breve introduzione sullo svolgimento del tirocinio, sull'applicazione LiveSignage e sua architettura, sullo stato dell'arte e sugli obiettivi raggiunti. Infine una descrizione della struttura di questa relazione.

% \chapter{Stato dell'Arte}

\textbf{Idea generale:} Descrizione dello stato attuale dell'applicazione, suoi usi e dispositivi con cui è compatibile.
\chapter{Tecnologie Utilizzate}

\section{In Breve}
Descrizione delle tecnologie a disposizione (due device diversi), loro differenze e scelte implementative da queste derivanti. Linguaggi e librerie utilizzate. 
\chapter{LiveSignage su WebOs}

\section{In Breve}
Descrizione del lavoro svolto, difficoltà incontrate e soluzioni trovate, modifiche all'applicazione pre-esistente. Descrizione della fase di testing finale.
\chapter{Studio di Funzionalità Aggiuntive}\label{aggiunte}

\section{In breve}
Una volta che l'applicazione è stata completata nella sua versione base ho avuto modo di aggiungere, o almeno studiare la possibilità di farlo, alcune funzionalità non presenti nella versione per dispositivi Samsung e che non erano previste a inizio tirocinio, ma che lo studio della documentazione ci ha portato a considerare.
In questo capitolo si presenta una descrizione della fase di analisi e, quando possibile, aggiunta delle nuove funzionalità e alcune idee per possibili miglioramenti futuri.

\section{Wake On Lan}

Nei dispositivi LG è implementato il protocollo Wake on LAN cioè la possibilità di accendere il dispositivo da remoto tramite l'invio di un messaggio, da parte di un device sulla stessa LAN, con protocollo UDP.
Il messaggio da inviare, chiamato magic packet, è formato da 102 bytes: i primi 6 rappresentano il numero 255, i successivi rappresentano l'indirizzo MAC del dispositivo ricevente \cite{wol}.

I device LG danno la possibilità di attivare questo protocollo sia in caso di connessione cablata che wireless, sono state implementate quindi le chiamate per verificare lo stato e attivare il protocollo dal back office.

In Listing \ref*{lst:wol} sono mostrate le funzioni per attivare e richiedere lo stato del protocollo in base al tipo di connessione del dispositivo.

\lstinputlisting[caption={Wake on LAN.}, label={lst:wol}, language=JavaScript]{listings/aggiunte/wol.js}

Tuttavia, dal momento che il dispositivo ricevente il messaggio deve trovarsi nella stessa rete del dispositivo mittente, non è stato possibile inserire l'accensione remota nel back office; sarà quindi compito del proprietario del dispositivo LG inviare il messaggio tramite un'opportuna applicazione per smartphone o per computer.

\section{Videochiamate}

Dal momento che i dispositivi LG supportano l'uso di webcam collegate tramite la porta USB, è stato fatto uno studio di fattibilità al fine di inserire un servizio di videochiamate all'interno dell'applicazione.
Questo plug-in è già presente sulla versione di Livesignage per dispositivi Raspberry e si basa su Jitsi, un progetto Open Source che implementa i protocolli e le interfacce necessarie \cite{jitsi}.

Sono stati fatti quindi alcuni test, che hanno dato esito positivo, per verificare la compatibilità delle telecamere e l'invio e la ricezione di video e audio. Tuttavia la RAM disponibile è risultata insufficiente per l'utilizzo di Jitsi Meet, l'applicazione per web call sviluppata da Jitsi, ed è stato quindi deciso di non proseguire lo studio dal momento che il tempo a disposizione non sarebbe risultato sufficiente.

\section{Possibili sviluppi futuri} \label{future_implementazioni}

Come detto nella presentazione delle API disponibili (capitolo \ref*{api}) la libreria \jscode{Harmony API} implementa i protocolli di comunicazione con alcuni device esterni:
\begin{itemize}
    \item Sensori near-field communication (NFC) che permettono lo scambio bidirezionale di messaggi in una rete peer-to-peer che si crea automaticamente quando due apparecchi NFC vengono a contatto \cite{nfc};
    \item Sensori radio-frequency identification (RFID) che, come per gli NFC, permettono lo scambio di messaggi tra sensori ed etichette tramite l'utilizzo di frequenze radio;
    \item Stampanti a calore tipicamente usate per gli scontrini. 
\end{itemize}

La possibilità di utilizzare questi dispositivi apre l'applicazione a diversi sviluppi futuri come pagamenti elettronici o la possibilità di mostrare contenuti diversi in base ai dati letti da uno dei sensori.

Il tempo a disposizione non ha permesso di andare oltre la fase di analisi di queste funzionalità ma sono stati comunque definiti i possibili protocolli per lo sviluppo.
\chapter{Conclusioni}

\section{Obiettivi Raggiunti}

\edit{Alla conclusione del tirocinio la versione di Livesignage per dispositivi basati su sistema operativo WebOS signage è stata completata in modo che soddisfacesse tutti i requisiti previsti:}

\edit{\begin{itemize}
    \item è stato mantenuto, nella maniera più ampia possibile, il codice HTML e CSS che si occupa dello stile dei contenuti da mostrare;
    \item l'applicazione è in grado di eseguire degli script JavaScript e, tramite l'uso di API specifiche, comunicare con il sistema operativo sottostante;
    \item viene stabilito un canale di comunicazione tramite socket con il web server su cui è presente il back office e in base ai messaggi ricevuti vengono modificati i contenuti mostrati o le impostazione del device;
    \item vengono creati e mantenuti i database, tramite l'utilizzo di file di testo, in modo che anche in caso di assenza di rete l'applicazione possa funzionare correttamente;
    \item è stata previsto il reindirizzamento di alcuni tasti del telecomando del dispositivo in modo da poter eseguire funzioni specifiche;
    \item sono state aggiunte le funzionalità per l'attivazione del protocollo WoL;
    \item è stato eseguito uno studio di fattibilità sia per l'aggiunta di un plug-in per le videochiamate sia per la comunicazione con eventuali device esterni.
\end{itemize}}

\edit{Non è stata trovata al momento una soluzione che porti una maggiore efficienza nella scrittura dei file nel file system del device, come mostrato in Sezione \ref*{database} sono state analizzate alcune possibilità ma nessuna è risultata applicabile al momento. }

\edit{Al momento l'applicazione è stata distribuita su alcuni dispositivi sia presso la sede dell'azienda, sia presso alcuni clienti ed è costantemente aggiornata insieme alle versioni per gli altri sistemi operativi.} 

\section{Competenze Acquisite}

\edit{Durante il tirocinio sono state sviluppate delle buone conoscenze riguardanti lo sviluppo di un'applicazione basata su linguaggi HTML, CSS e JavaScript, sulle tecnologie e i protocolli utilizzati nelle applicazioni web e sull'utilizzo delle risorse hardware di sistema. 
È stata inoltre sviluppata l'attenzione per quanto riguarda l'esperienza utente e alla manutenibilità del codice.} 




\bibliographystyle{unsrt}
\bibliography{chapters/Bibliografia.bib}


\end{document}
% -----------------------------------------------------------------
