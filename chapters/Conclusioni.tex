\chapter{Conclusioni}

\section{Obiettivi Raggiunti}

\edit{Alla conclusione del tirocinio la versione di Livesignage per dispositivi basati su sistema operativo WebOS signage è stata completata in modo che soddisfacesse tutti i requisiti previsti:}

\edit{\begin{itemize}
    \item è stato mantenuto, nella maniera più ampia possibile, il codice HTML e CSS che si occupa dello stile dei contenuti da mostrare;
    \item l'applicazione è in grado di eseguire degli script JavaScript e, tramite l'uso di API specifiche, comunicare con il sistema operativo sottostante;
    \item viene stabilito un canale di comunicazione tramite socket con il web server su cui è presente il back office e in base ai messaggi ricevuti vengono modificati i contenuti mostrati o le impostazione del device;
    \item vengono creati e mantenuti i database, tramite l'utilizzo di file di testo, in modo che anche in caso di assenza di rete l'applicazione possa funzionare correttamente;
    \item è stata previsto il reindirizzamento di alcuni tasti del telecomando del dispositivo in modo da poter eseguire funzioni specifiche;
    \item sono state aggiunte le funzionalità per l'attivazione del protocollo WoL;
    \item è stato eseguito uno studio di fattibilità sia per l'aggiunta di un plug-in per le videochiamate sia per la comunicazione con eventuali device esterni.
\end{itemize}}

\edit{Non è stata trovata al momento una soluzione che porti una maggiore efficienza nella scrittura dei file nel file system del device, come mostrato in Sezione \ref*{database} sono state analizzate alcune possibilità ma nessuna è risultata applicabile al momento. }

\edit{Al momento l'applicazione è stata distribuita su alcuni dispositivi sia presso la sede dell'azienda, sia presso alcuni clienti ed è costantemente aggiornata insieme alle versioni per gli altri sistemi operativi.} 

\section{Competenze Acquisite}

\edit{Durante il tirocinio sono state sviluppate delle buone conoscenze riguardanti lo sviluppo di un'applicazione basata su linguaggi HTML, CSS e JavaScript, sulle tecnologie e i protocolli utilizzati nelle applicazioni web e sull'utilizzo delle risorse hardware di sistema. 
È stata inoltre sviluppata l'attenzione per quanto riguarda l'esperienza utente e alla manutenibilità del codice.} 

