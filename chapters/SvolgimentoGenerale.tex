\chapter{Livesignage su WebOs}\label{svolgimento}
\section{In Breve}
Descrizione del lavoro svolto: \edit{portare l'applicazione già esistente per i dispositivi Samsung sui dispositivi LG}, difficoltà incontrate e soluzioni trovate, modifiche all'applicazione pre-esistente. Descrizione della fase di testing finale.

\section{Fase di analisi}

Nelle prime fasi del tirocinio il lavoro ha riguardato l'analisi dell'applicazione preesistente, della documentazione delle interfacce messe a disposizione da LG e lo studio dei linguaggi di programmazione utilizzati (vedi \ref*{linguaggi}).

Nella Fig.\ref*{fig:architettura_2} è presentato uno schema dell'architettura dell'applicazione lato client in cui sono mostrate le componenti principali: Livesignage, il sistema operativo e il web server che ospita il back office.

\begin{itemize}
    \item Web server: qui vengono salvati in un database MySQL tutte le informazioni degli utenti, dei dispositivi e le playlist create, tutte queste informazioni vengono elaborate e inviate ai client attraverso una socket;
    \item Livesignage: dopo aver aperto stabilito il canale di comunicazione con il web server resta in attesa dei messaggi da questo inviati e si occupa di svolgere le operazioni richieste comunicando, tramite le interfacce di sistema, con il sistema operativo sottostante.
    \item \tomodify{Sistema operativo: riceve le richieste da parte dell'applicazione e una volta effettuate le operazioni chiama una funzione di callback definita precedentemente su cui è inviata la risposta.}
\end{itemize}

Gli altri componenti del sistema sono: il database principale su cui viene salvato il codice HTML delle playlist, il database dei log in cui sono scritti i messaggi di errore in modo che siano successivamente \tomodify{scaricabili} dal creatore di contenuti e la cartella degli assets nella quale vengono salvate le immagini e i video delle playlist create in modo che queste possano essere visualizzate anche offline.

\subsection{Codice Livesignage}

Di seguito è presentata brevemente la struttura dell'applicazione.

\subsubsection{Suddivisione del codice}

Nella cartella principale sono presenti 3 file HTML: 
\begin{itemize}
    \item index.html: è la prima pagina ad essere caricata e si occupa di chiamare le funzioni Javascript per controllare se il device sia stato precedentemente associato a Livesignage.
    \item noAssociation.html: questa pagina viene caricata nel caso che il dispositivo non sia associato e mostra un codice univoco per l'associazione.
    \item display.html: questa è la pagina in cui viene iniettato il codice delle playlist e dei plugin.
\end{itemize}

Il resto dei file fondamentali si trova nella cartella js, qui sono raccolti gli script e le classi necessarie al corretto svolgimento dell'applicazione:

\begin{itemize}
    \item association.js: classe che controlla e gestisce l'associazione del device;
    \item connection.js: i metodi di questa classe vengono chiamati ciclicamente per verificare lo stato della connessione;
    \item db.js: classe che si occupa della gestione del database principale, verrà vista in maniera più approfondita ( insieme a dblog.js ) in \ref*{database};
    \item dblog.js: qui è contenuta la gestione del database degli errori;
    \item display.js: in questa classe sono definiti i metodi che implementano la comunicazione con il sistema operativo attraverso le API. Alcune di queste implementazioni verranno \tomodify{evidenziate} in \ref*{webos_doc};
    \item filemanager.js: si occupa della gestione del file system del sistema e del download e della cancellazione degli assets;
    \item livesignage.js: in questa classe sono contenuti i metodi per la visualizzazione delle playlist, la gestione di alcuni comandi ricevuti dal server come il cambio di slide e il passaggio alla visualizzazione dell'ingresso HDMI e le richieste di salvataggio e cancellazione delle playlist dal database;
    \item log.js: qui sono presenti i metodi per la creazione dei messaggi di errore e l'invio di questi al back office in modo che possano essere visualizzati dal cliente;
    \item main.js: questo è lo script principale che si occupa di chiamare le altre classi, in particolare controlla se sia necessario scaricare un aggiornamento dell'applicazione o del firmware del sistema operativo (vedi \ref*{update}), chiama a intervalli di tempo i metodi per il controllo della connessione, invoca i metodi che si occupano della comunicazione con il back office e, se il display risulta offline, avvia la visualizzazione dell'ultima playlist salvata.
    \item monitor.js: questa classe si occupa di aprire la socket verso il back office e di restare in ascolto su questa, alla ricezione dei messaggi chiama i metodi necessari.
    \item speedtest.js: questa classe viene invocata quando si riceve una richiesta per valutare lo stato della rete dal back office.
\end{itemize}

\subsection{Documentazione WebOs Signage}\label{webos_doc}

Dopo l'analisi dell'applicazione preesistente è stata studiata la documentazione (disponibile in \cite{LgDoc}) dei dispositivi LG.

Come detto in \ref*{api} sono state analizzate le diverse interfacce messe a disposizone dal sistema per la comunicazione tra il browser e il sistema operativo; le librerie disponibili sono: 
\begin{itemize}
    \item SCAP API: un set di interfacce specifiche per WebOs Signage, la versione di WebOs dedicata ai sistemi di digital signage, \tomodify{suddivise in diverse classi a seconda del loro uso}. Questo è il set scelto per lo sviluppo poiché, sebbene non sia il più recente, è l'unico compatibile con tutti i device di digital signage di LG e che abbia un set ampio di interfacce tale da coprire i requisiti dell'applicazione;
    \item IDCAP API: è il nuovo standard di API di LG che ha come obbiettivo quello di rendere univoche le interfacce per le diverse versioni di WebOs sia quella per il digital signage che quella per le Tv commerciali. Nonostante questo set sia quello raccomandato da LG è supportato solamente da WebOs Signage 6.0 e successivi;
    \item Harmony API: una libreria, specifica per WebOs Signage, contenente le interfacce per la comunicazione con alcuni dispositivi esterni come ad esempio stampanti, sensori NFC e infrarossi;
    \item CustomJS API: questo set aggiunge alcune interfacce mancanti alla libreria SCAP. Questa libreria è aggiunta all'applicazione sviluppata al fine di espanderne le funzionalità (si veda \ref*{aggiunte}).
\end{itemize}

Sono state anche studiate le informazioni hardware e software dei dispositivi cercando di capire quali fossero \tomodify{le possibilità} minime comuni a tutti. Dal momento che le componenti software e hardware sono molto diverse tra loro sono stati presi in considerazione solo i dispositivi che sopportano almeno WebOs signage 4.0. Di seguito sono riportate le specifiche e le versioni dei linguaggi di programmazione supportate che sono state definite dopo l'analisi.

\begin{center}
\begin{tabular}{ |l|r| } 
     \hline
     HTML & versione 5 \\ 
     CSS & versione 3 \\ 
     JavaScript & versione 1.6+ \\ 
     \hline
     Web Engine & Chrome 53 \\ 
     HTTP, HTTPS & \checkmark \\ 
     XMLHttpRequest (AJAX) & \checkmark \\ 
     JSON & \checkmark \\ 
     \hline
     RAM & 2.0 GB \\
     Memoria & 8.0 GB\\
     Risoluzione & 1920x1080\\
     \hline
\end{tabular}
\end{center}

\section{Scrittura delle classi JavaScript}

\lstinputlisting[language=JavaScript, firstline=0, lastline=18]{listings/svolgimento/api_sample.js}
\section{Cambiamenti di logiche e scelte implementative}
\subsection{Lettura e scrittura dei database} \label{database}
\subsection{Cancellazione del localStorage}
\subsection{Gestione della connessione}
\subsection{Gestione dei timer}
\subsection{Aggiornamento del firmware di sistema e dell'applicazione}\label{update}
\subsection{Gestione input del telecomando}

\section{Fase di testing}


