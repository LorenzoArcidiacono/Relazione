\chapter{Introduzione}
\linespread{1.5}

\section{In Breve}
Breve introduzione sullo scopo e svolgimento del tirocinio compreso il contesto in cui ho svolto il lavoro, sull'applicazione LiveSignage e sua architettura, sullo stato dell'arte e sugli obiettivi raggiunti. Infine una descrizione della struttura di questa relazione.

\section{Scopo del Tirocinio}

\edit{Il tirocinio è stato svolto presso l'azienda Softhrod srl. di Volterra all'interno del progetto LiveSignage, un'applicazione focalizzata sulla distribuzione di contenuti multimediali targettizzati. Scopo del tirocinio è l'analisi, la progettazione e lo sviluppo di un'applicazione lato client per il sistema operativo WebOs Signage sui dispositivi LG:  display professionali, videowall, smart TV, sistemi IoT, media player. L'applicazione deve essere in grado di mostare specifici contenuti multimediali sulla base di eventi acquisiti dall'ambiente circostante, come ad esempio condizioni metereologiche o orario, e parametri indicati dall'utente. L'applicazione è stata sviluppata in linguaggio Javascript sfruttando le API REST messe a disposizione dal sistema operativo.}

\section{Modalità di Lavoro}

Ho lavorato in modalità mista ( telematica e presenza ) in maniera prevalentemente autonoma, ma mantenendo un confronto diretto sia con gli altri sviluppatori sia con i designer per tutte le scelte riguardanti le logiche e lo sviluppo dell'applicazione. Ho così avuto modo di capire quale sia il processo produttivo dell'azienda, come si lavora in team e l'attenzione per l'esperienza utente, fondamentale per questo software.

\section{Digital signage e LiveSignage}

\edit{Il digital signage è una forma di comunicazione di prossimità sul punto vendita o in spazi pubblici aperti o all'interno di edifici, anche nota in Italia come segnaletica digitale, videoposter o cartellonistica digitale, i cui contenuti vengono mostrati ai destinatari attraverso schermi elettronici o videoproiettori appositamente sistemati in luoghi pubblici. }

LiveSignage è un’applicazione per il digital signage sviluppata per vari dispositivi specializzati tra i quali quelli di Samsung. L’idea è quella di permettere al cliente di aumentare il proprio engagement in modo semplice, permettendo di mostrare facilmente qualsiasi contenuto: filmati, foto, infografiche, webfeed e informazioni in tempo reale. Tramite l'utilizzo di alcuni plugin, è possibile ampliare ancora i casi d'uso del servizio: come la possibilità di fare webcall e l'integrazione di social network.

\begin{figure}[!htb]
    \centering
    \includegraphics[width= 0.5\textwidth]{images/Introduzione/LiveTurist.jpg} 
    \caption{Esempio di infografica in un'installazione cittadina.} 
\end{figure}

Inoltre, tramite un codice QR creato automaticamente insieme ai contenuti mostrati, l'utente finale può ricevere le informazioni in modo più dettagliato sul proprio dispositivo mobile senza bisogno di scaricare applicazioni o fare ricerche online.

I casi d'uso sono molteplici: cinema, ristoranti, turismo, sanità e soluzioni personalizzate, per fare alcuni esempi.

\section{Stato dell'arte}

All'inizio del tirocinio LiveSignage era già disponibile per dispositivi di Samsung e Raspberry.  

L'idea alla base dell'applicazione è quella di mostrare una serie di slide organizzate in playlist; ogni slide può mostrare diversi tipi di contenuti come ad esempio immagini, video, web page, mappe, notizie, input esterni. Le playlist possono essere di diverso tipo:

\begin{itemize}
    \item Semplici: le slide vengono mostrate a schermo intero e si susseguono continuamente;
    \item Composte: in questo caso lo spazio a disposizione viene suddiviso in più parti e in ognuna di queste viene mostrata una playlist diversa;
    \item Concatenate: vengono concatenate più playlist una dopo l'altra.
    \begin{figure}[!htb]
        \centering
        \includegraphics[width= 0.8\textwidth]{images/Introduzione/playlist-composta.png} 
        \caption{Una playlist composta.} 
    \end{figure}
\end{itemize}


Oltre alle playlist è possibile utilizzare dei plugin per poter dare al cliente più possibilità di personalizzazione. Ad esempio permettendo di  mostrare contenuti interattivi utilizzando l'input dell'utente tramite il touchscreen o di inserire il proprio e-commerce.

\subsection{Architettura}
\begin{figure}[!htb]
    \centering
    \includegraphics[width= 0.8\textwidth]{images/Introduzione/architettura.jpg} 
    \caption{Architettura dell'applicazione.} 
    \label{fig:architettura}
\end{figure}


L'applicazione è installata direttamente sul device, il cliente può impostare le playlist o eventuali modifiche del sistema come: timer di accensione e spegnimento, fuso orario, volume e luminosità, direttamente da un browser o dall'applicazione per dispositivi mobili (Figura: \ref*{fig:schermata-web}).\nl
Queste modifiche vengono inviate, tramite chiamate API al BackOffice che attraverso una socket si occupa di comunicare all'applicazione i cambiamenti avvenuti. 
L'applicazione utilizza le API del device per settare le modifiche di sistema e scrivere i file necessari per il mantenimento dei database e il salvataggio degli assets necessari: immagini e video. \n
Sono utilizzati due database: il primo è usato per salvare tutte le informazioni sulle playlist inserite fino a quel momento, eventuali timer, ID del dispositivo e altre informazioni utili al corretto funzionamento, il secondo salva i messaggi, tipicamente di errore, dell'applicazione che possono essere successivamente scaricati.\n
Per diminuire i tempi di attesa dovuti ai download e il carico della rete le immagini e i video scaricati vengono salvati sul dispositivo e non saranno cancellate a meno di una richiesta specifica del cliente o in caso di necessità di liberare memoria.\n

La figura \ref*{fig:architettura} mostra in maniera schematica l'architettura dell'applicazione e l'interazione con l'utente finale.

\begin{figure}[!htb]
    \centering
    \includegraphics[width= 1\textwidth]{images/Introduzione/SchermataWebLS.jpg} 
    \caption{Schermata della Web Page per programmare il display.} 
    \label{fig:schermata-web}
\end{figure}

\section{Obiettivi raggiunti}

Al termine del tirocinio la versione di LiveSignage per WebOS Signage è stata correttamente sviluppata; ho avuto inoltre modo di aggiungere, o almeno studiare la possibilità di farlo alcune funzionalità non presenti nella versione per dispositivi Samsung, le quali non erano previste a inizio tirocinio, che lo studio della documentazione ci ha portato a considerare.

\section{Struttura della Relazione}
Di seguito una breve descrizione della suddivisione in capitoli di questa relazione.
\begin{itemize}
    \item Tecnologie utilizzate: Descrizione delle tecnologie a disposizione, loro differenze e scelte implementative da queste derivanti. Linguaggi e librerie utilizzate.
    \item LiveSignage su WebOs: Descrizione del lavoro svolto, difficoltà incontrate e soluzioni trovate, modifiche all'applicazione pre-esistente. Descrizione della fase di testing finale.
    \item Studio di funzionalità aggiuntive: In questo capitolo si presenta una descrizione della fase di analisi e, quando possibile, aggiunta delle nuove funzionalità e alcune idee per possibili miglioramenti futuri.
    \item Conclusioni: Riassunto dei punti principali affrontati e degli obiettivi raggiunti. Competenze acquisite.
\end{itemize}