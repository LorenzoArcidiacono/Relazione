\chapter{Introduzione}
\linespread{1.5}

\section{In Breve}
Breve introduzione sullo \edit{scopo e } svolgimento del tirocinio \edit{compreso il contesto in cui ho svolto il lavoro}, sull'applicazione LiveSignage e sua architettura, sullo stato dell'arte e sugli obiettivi raggiunti. Infine una descrizione della struttura di questa relazione.

\section{Scopo del Tirocinio}

Il tirocinio è stato svolto presso Softhrod srl. all'interno del progetto LiveSignage, focalizzato sulla distribuzione di contenuti multimediali targettizzati (client profile, store location, weather, product RFID etc.), dove ho partecipato alle fasi di analisi, progettazione e sviluppo della soluzione atta a sfruttare i dispositivi LG: professional display, videowall, smart TV, sistemi IoT, media player. Entrando a far parte del team di sviluppo e confrontandomi con UI/UX designer, sviluppatori front-end e back-end.

L'obiettivo del lavoro era quello di sviluppare una versione dell'applicazione LiveSignage, capace di visualizzare specifici contenuti multimediali sulla base di eventi acquisiti e parametri indicati dall'utente, sviluppata in linguaggio Javascript sfruttando le API \tomodify{REST} della piattaforma, per il sistema operativo di LG: WebOS Signage.

\section{Modalità di Lavoro}

Ho lavorato in modalità mista ( telematica e presenza ) in maniera autonoma, ma mantenendo un confronto diretto sia con gli altri sviluppatori sia con i designer per tutte le scelte delle logiche e di sviluppo dell'applicazione. \tomodify{Ho così avuto modo di capire quale sia il processo produttivo dell'azienda, come si lavora in team e l'attenzione per l'esperienza utente.}

\section{LiveSignage}

LiveSignage è un’applicazione per il digital signage sviluppata per vari dispositivi specializzati tra i quali quelli di Samsung. L’idea è quella di permettere al cliente di aumentare il proprio engagement in modo semplice, permettendo di mostrare facilmente qualsiasi contenuto: filmati, foto, infografiche, webfeed e informazioni in tempo reale.

\begin{figure}[!htb]
    \centering
    \includegraphics[width= 0.5\textwidth]{images/Introduzione/LiveTurist.jpg} 
    \caption{LiveSignage per scopi turistici.} 
\end{figure}

Inoltre, tramite un codice QR creato automaticamente insieme ai contenuti mostrati, l'utente finale può ricevere le informazioni in modo più dettagliato sul proprio dispositivo mobile senza il bisogno di scaricare applicazioni o fare ricerche online.

\tomodify{I casi d'uso sono molteplici: cinema, ristoranti, turismo, sanità o soluzioni personalizzate per fare alcuni esempi.
Tramite l'utilizzo di alcuni plugin è inoltre possibile ampliare ancora le possibilità del servizio come la possibilità di fare webcall e l'integrazione di social network.}

\subsection{Architettura}
\begin{figure}[!htb]
    \centering
    \includegraphics[width= 0.8\textwidth]{images/Introduzione/architettura.jpg} 
    \caption{Architettura dell'applicazione.} 
    \label{fig:architettura}
\end{figure}

\tomodify{immagine dell'architettura da modificare: \url{https://miro.com/app/board/uXjVOjz-X_4=/}}

\tomodify{L'applicazione è installata direttamente sul device,il cliente può selezionare cosa far visualizzare ed eventuali modifiche del sistema come timer di accensione e spegnimento, fuso orario, volume e luminosità direttamente da un browser o dall'applicazione (Figura: \ref*{fig:schermata-web}).\nl
Queste modifiche vengono inviate, tramite chiamate API all'BackOffice che attraverso una socket si occupa di comunicare all'applicazione i cambiamenti avvenuti. 
L'applicazione utilizza le API del device per settare le modifiche di sistema e scrivere i file necessari per il mantenimento del database. Tale database è usato per salvare tutte le informazioni sui contenuti mostrati fino a quel momento, eventuali timer, ID del dispositivo e altre informazioni utili al corretto funzionamento. 
Le immagini e i video scaricati vengono salvati sul dispositivo e non saranno cancellate a meno di una richiesta specifica del cliente o in caso di necessità di liberare memoria. Questo per evitare di scaricare più volte lo stesso file.}
\begin{figure}[!htb]
    \centering
    \includegraphics[width= 1\textwidth]{images/Introduzione/SchermataWebLS.jpg} 
    \caption{Schermata della pagina web del device.} 
    \label{fig:schermata-web}
\end{figure}
\section{Stato dell'arte}
\section{Obiettivi raggiunti}
\section{Struttura della Relazione}