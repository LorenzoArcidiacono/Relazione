\chapter{Introduzione}
\linespread{1.5}

\section{In Breve}
Breve introduzione sullo svolgimento del tirocinio, sull'applicazione LiveSignage e sua architettura, sullo stato dell'arte e sugli obiettivi raggiunti. Infine una descrizione della struttura di questa relazione.

\section{Scopo del Tirocinio}

Il tirocinio è stato svolto presso Softhrod srl. all'interno del progetto LiveSignage, focalizzato sulla distribuzione di contenuti multimediali targettizzati (client profile, store location, weather, product RFID etc.), dove ho partecipato alle fasi di analisi, progettazione e sviluppo della soluzione atta a sfruttare i dispositivi LG: professional display, videowall, smart TV, sistemi IoT, media player. Entrando a far parte del team di sviluppo e confrontandomi con UI/UX designer, sviluppatori front-end e back-end.

L'obiettivo del lavoro era quello di sviluppare una versione dell'applicazione LiveSignage, capace di visualizzare specifici contenuti multimediali sulla base di eventi acquisiti e parametri indicati dall'utente, sviluppata in linguaggio Javascript sfruttando le API \torevise{REST} della piattaforma, per il sistema operativo di LG: WebOS Signage.

\section{LiveSignage}

LiveSignage è un’applicazione per il digital signage sviluppata per vari dispositivi specializzati tra i quali quelli di Samsung. L’idea è quella di permettere al cliente di aumentare il proprio engagement in modo semplice, permettendo di mostrare facilmente qualsiasi contenuto: filmati, foto, infografiche, webfeed e informazioni in tempo reale.
Inoltre, tramite un codice QR creato automaticamente insieme ai contenuti mostrati, l'utente può ricevere le informazioni in modo più dettagliato sul proprio smartphone senza il bisogno di scaricare applicazioni o fare ricerche online.

\begin{figure}[!htb]
    \label{fig:architettura}
    \centering
    \includegraphics[width= 0.8\textwidth]{images/Introduzione/architettura.jpg} 
    \caption{Architettura dell'applicazione} 
\end{figure}

\section{Stato dell'arte}
\section{Obiettivi raggiunti}
\section{Struttura della Relazione}