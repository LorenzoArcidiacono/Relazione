\chapter{Tecnologie utilizzate}

\section{In breve}
Descrizione delle tecnologie a disposizione (due device diversi), loro differenze e scelte implementative da queste derivanti. Linguaggi e librerie utilizzate. 

\section{Dispositivi a disposizione}
Per lo svolgimento del tirocinio sono stati messi a disposizione due device LG con versioni differenti del sistema operativo:
\begin{itemize}
    \item Un player (un device non munito di uno schermo) con la versione 4.0;
    \item Un professional display con la versione 6.0.
\end{itemize}

Oltre alle differenze hardware i due modelli presentano importanti differenti al livello del software; sul primo è installato il web engine Chromium 53 e supporta solamente la visualizzazione di 2 video contemporaneamente, il secondo supporta fino a 4 video e ha installato Chromium 79.

La differenza di web engine ha comportato un'analisi approfondita delle librerie e delle funzioni JavaScript supportate e ha quindi guidato le scelte implementative del progetto.


\section{Linguaggi di programmazione e librerie}

Il tirocinio, oltre al linguaggio di markup HTML e quello di stile CSS, si è basato sul linguaggio di programmazione JavaScript: un linguaggio interpretato usato principalmente nella programmazione front-end delle pagine web (la cui documentazione è reperibile in \cite{MdN}).

Oltre alle librerie di sistema (vedi \ref{api}), sono state utilizzate la librerie JQuery \cite{jQDoc} per la manipolazione delle pagine html e per la gestione degli eventi e la libreria Dixie.js \cite{dixie} per la gestione dei database (descritti in \ref*{database}). Entrambe le librerie sono già utilizzate negli altri client sviluppati dall'azienda ed è stato quindi deciso di mantenerle sia per una questione di manutenibilità dei diversi client, sia perché comportano notevoli miglioramenti alla leggibilità del codice e al suo funzionamento.

\tomodify{Non avendo familiarità con i linguaggi e le tecnologie utilizzati si è reso necessario un'iniziale studio approfondito dei linguaggi e delle librerie necessarie.}

\subsection{Scelta delle API} \label{api}

\tomodify{Dal momento che il linguaggio Javascript, per questioni di sicurezza, non ha modo di comunicare con il sistema operativo sottostante in modo nativo, sono state messe a disposizione, dagli sviluppatori LG, delle interfacce implementate sottoforma di librerie Javascript per poter eseguire operazioni e ricevere informazioni dal device quali: informazioni di rete, stato del dispositivo, eventuali timer di accensione e spegnimento e gestione degli aggiornamento dell'applicazione e del sistema operativo.}

I set di API messi a disposizione sono molteplici (come specificato in \cite{LgDoc}), tuttavia non tutti sono compatibili con tutte le versioni del sistema operativo. È stato quindi deciso di basare il progetto sul set SCAP (Signage Application Platform) API v1.8 poiché questo è supportato dalla maggior parte dei dispositivi LG in commercio. Alcune di queste interfacce saranno approfondite nel capito \ref*{svolgimento}.

Non tutte le API messe a disposizione sono state utilizzate durante lo svolgimento del tirocinio, è stata comunque fatta un'analisi al fine di capire quali possano essere di interesse per implementazioni successive (vedi \ref*{future_implementazioni}).