\chapter{Tecnologie Utilizzate}

\section{In Breve}
Descrizione delle tecnologie a disposizione (due device diversi), loro differenze e scelte implementative da queste derivanti. Linguaggi e librerie utilizzate. 

\section{Dispositivi a Disposizione}
Per lo svolgimento del tirocinio sono stati messi a disposizione due device LG con versioni differenti del sistema operativo:
\begin{itemize}
    \item Un player (un device non munito di uno schermo) con la versione 4.0;
    \item Un professional display con la 6.0.
\end{itemize}

Oltre alle differenze hardware i due modelli presentano importanti differenti al livello del software; sul primo è installato il web engine Chromium 53 e supporta solamente la visualizzazione di 2 video contemporaneamente, il secondo supporta fino a 4 video e ha installato Chromium 79.

La differenza di web engine ha comportato un'analisi approfondita delle librerie e delle funzioni JavaScript supportate e ha quindi guidato le scelte implementative del progetto.

\subsection{Scelta delle API}

\tomodify{Dal momento che Javascript non ha modo di comunicare con il sistema operativo sottostante sono state create, dagli sviluppatori LG, delle interfacce, implementate come libreria Javascript per poter eseguire operazioni e ricevere informazioni dal device.}

\tomodify{Queste interfacce permettono di ricevere e impostare molte informazioni quali: stato della connessione, eventuali timer di accensione e spegnimento, scrittura e lettura di file}

I set di API messi a disposizione sono molteplici (come specificato in \cite{LgDoc}), tuttavia non tutti sono compatibili con tutte le versioni del sistema operativo. È stato quindi deciso di basare il progetto sul set SCAP (Signage Application Platform) API v1.8 poichè questo è supportato dalla maggior parte dei dispositivi in commercio.


\section{Linguaggi di Programmazione e Librerie}

Per lo svolgimento del tirocinio, oltre al linguaggio di markup HTML e quello di stile CSS, è stato usato il linguaggio di programmazione JavaScript: un linguaggio interpretato usato principalmente nella programmazione front-end delle pagine web (la documentazione è reperibile in \cite{MdN}).

Sono state utilizzate le librerie JQuery \cite{jQDoc} per la manipolazione delle pagine html e per la gestione degli eventi e Dixie.js \cite{dixie} per la gestione dei database.

\tomodify{Si è cercato di mantenere, per quanto possibile, l'utilizzo delle librerie usate per i client degli altri sistemi operativi in modo da rendere l'applicazione facilmente mantenibile.}

\tomodify{Non avendo familiarità con i linguaggi e le tecnologie utilizzati si è reso necessario un'iniziale studio approfondito dei linguaggi e delle librerie necessarie.}